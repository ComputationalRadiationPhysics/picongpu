\documentclass[a4paper,12pt]{article}
\usepackage[ngerman]{babel}
\pagenumbering{arabic}
\usepackage{ae}
\usepackage[utf8x]{inputenc}
\usepackage{amsmath}

\title{Vay-Pusher}
\author{Richard Pausch}
\date{12. June 2012}

\begin{document}

\section{Vay-Pusher}

The Vay Pusher is a particle pusher described in ''Simulation of beams or plasmas crossing at relativistic velocity`` by J.-L- Vay, promising to overcome problems with the famous Boris-Pusher at relativistic velocities. The formulas of the paper have been adjusted to PIConGPU and are summarized here. For more details please refer to the original paper.


\subsection{Conventions}
All indices have been reduced by a $\frac{1}{2}$-step compared to the paper.
Locations ($\vec r_i$) and fields ($\vec E_i$, $\vec B_i$) are known at full time steps, momenta and speeds ($\vec p_i$, $\vec u_i$, $\vec v_i$) at half time steps. In contrast to the Vay paper real momenta ($\vec p_i = \vec u_i \cdot m_0$) will be used.
\begin{eqnarray*}
\mbox{Vay paper} & \rightarrow & \mbox{implematation in PIC} \\
\vec u^{i} & \rightarrow & \vec p_{-1/2} \mbox{\quad  (known at start)}\\
\vec u^{i+1/2} & \rightarrow & \vec p_{0}\\
\vec u^{i+1} & \rightarrow & \vec p_{+1/2}\\
\vec E^{i+1/2} & \rightarrow & \vec E_{0}  \mbox{\quad  (known at start)}\\
\vec B^{i+1/2} & \rightarrow & \vec B_{0}  \mbox{\quad  (known at start)}\\
\vec r^{i+1/2} & \rightarrow & \vec r_{0}  \mbox{\quad  (known at start)}
\end{eqnarray*}

\subsection{Variables handed to particle pusher}
The pusher takes:
\begin{itemize}
	\item $\vec E_0$ \quad \texttt{eField}
	\item $\vec B_0$ \quad \texttt{bField}
	\item $\vec r_0$ \qquad \texttt{pos}
	\item $\vec p_{-1/2}$ \quad \texttt{mom}
\end{itemize}


\subsection{First step of the algorithm}
\begin{eqnarray}
\vec v_{i} & = & \frac{\vec p_{i}}{\sqrt{m^2 + \frac{{\vec p_i}^2}{c^2}}} \\
\vec p_{0} & = & \vec p_{-1/2} + \frac{q \Delta t}{2} \left(  \vec E_{0} + \vec v_{-1/2} \times \vec B_{0} \right)
\end{eqnarray}

\subsection{Second step of the algorithm}
\begin{eqnarray}
\vec p \prime  & = & \vec p_{0} + \frac{q \Delta t}{2} \vec E_{0} \\
\gamma \prime & = & \sqrt{1 + \frac{{\vec p \prime}^2}{m_0^2 c^2 }} \\
\vec \tau & = & \frac{q \Delta t}{2 m_0} \vec B_{0} \\
u^{\star} & = & \frac{\vec p \prime \cdot \vec \tau}{m_0 c} \\
\sigma & = & {\gamma \prime }^2 - {\vec \tau}^2 \\
\gamma_{+1/2} & = & \sqrt{\frac{  \sigma + \sqrt{  \sigma^2 + 4 \left( {\vec \tau}^2 + {u^\star}^2 \right) } }{2}} \\
\vec t & = & \frac{\vec \tau}{\gamma_{+1/2}} \\
s & = & \frac{1}{1 + \vec t^2} \\
\vec p_{+1/2} & = & s \cdot \left\{   \vec p \prime + \left( \vec p \prime \cdot \vec t  \right) \cdot \vec t   + \vec p \prime \times \vec t  \right\} \mbox{\quad  new momentum}
\end{eqnarray}

\subsection{Particle push}
The push of the particle is the same as used in the Boris-Pusher.



\end{document}
